%% ================================================================================
%% This LaTeX file was created by AbiWord.                                         
%% AbiWord is a free, Open Source word processor.                                  
%% More information about AbiWord is available at http://www.abisource.com/        
%% ================================================================================

\documentclass[a4paper,portrait,12pt]{article}
\usepackage[latin1]{inputenc}
\usepackage{calc}
\usepackage{setspace}
\usepackage{fixltx2e}
\usepackage{graphicx}
\usepackage{multicol}
\usepackage[normalem]{ulem}
\usepackage{color}
\usepackage{hyperref}
 
\begin{document}


\begin{flushleft}
Estudio de Geodésicas a través de Análisis
\end{flushleft}


\begin{flushleft}
Numérico
\end{flushleft}


\begin{flushleft}
Francisco de Izaguirre (4.425.135-0),
\end{flushleft}


\begin{flushleft}
Francisco Fernandez (4.596.080-9),
\end{flushleft}


\begin{flushleft}
Gabriela Mullukian (5.121.434-9), Marco Rolón (4.916.721-9)
\end{flushleft}


\begin{flushleft}
Instituto de Matemática y Estad\i{}́stica
\end{flushleft}


\begin{flushleft}
Facultad de Ingenier\i{}́a. Universidad de la República
\end{flushleft}


\begin{flushleft}
Montevideo, Uruguay
\end{flushleft}





\begin{flushleft}
Abstract
\end{flushleft}


\begin{flushleft}
El camino más corto entre dos puntos en un plano, es el segmento de recta que los une. Si
\end{flushleft}


\begin{flushleft}
se quiere analizar qué pasa en superficies curvas el problema deja de ser trivial. Dada una
\end{flushleft}


\begin{flushleft}
superficie y dos puntos cualesquiera contenidos en ella, se define la curva geodésica como
\end{flushleft}


\begin{flushleft}
la curva de menor longitud, de dicha superficie, que une esos dos puntos.
\end{flushleft}


\begin{flushleft}
En este trabajo se realiza una introducción al problema, una descripción matemática del
\end{flushleft}


\begin{flushleft}
mismo. Se demuestra que puede ser modelado como un PVI. Luego se utiliza el Método
\end{flushleft}


\begin{flushleft}
de Euler Hacia Adelante para su resolución. Posteriormente se presenta los resultados
\end{flushleft}


\begin{flushleft}
obtenidos, se realiza un análisis de los errores cometidos.
\end{flushleft}


\begin{flushleft}
CONCLUSION BREVE
\end{flushleft}


\begin{flushleft}
Keywords: Geodésicas, Métodos Numéricos, Euler hacia Adelante.
\end{flushleft}





1





\begin{flushleft}
Emails:\{francisco.de.izaguirre,francisco.fernandez,
\end{flushleft}


\begin{flushleft}
maria.mullukian,marco.rolon\}@fing.edu.uy
\end{flushleft}





\begin{flushleft}
\newpage
Contents
\end{flushleft}


\begin{flushleft}
1 Introducción
\end{flushleft}


\begin{flushleft}
2 Problema
\end{flushleft}


\begin{flushleft}
2.1 Ecuaciones diferenciales que rigen una geodésica
\end{flushleft}


\begin{flushleft}
2.2 Cálculo de las ecuaciones en superficie paramétrica
\end{flushleft}


\begin{flushleft}
3 Metodolog\i{}́a
\end{flushleft}


\begin{flushleft}
3.1 Problema de Valor Inicial (PVI)
\end{flushleft}


\begin{flushleft}
3.2 Discretización del problema
\end{flushleft}


\begin{flushleft}
4 Estudio Experimental
\end{flushleft}


\begin{flushleft}
4.1 Ambiente de trabajo
\end{flushleft}


\begin{flushleft}
4.2 Resultados Obtenidos
\end{flushleft}


\begin{flushleft}
5 Conclusiones
\end{flushleft}


\begin{flushleft}
References
\end{flushleft}





3


5


5


10


14


14


15


17


17


17


19


20





\newpage
1





\begin{flushleft}
Introducción
\end{flushleft}





\begin{flushleft}
Trasladarse desde un punto a otro sobre una superficie recorriendo el camino más
\end{flushleft}


\begin{flushleft}
corto es un problema tan añejo como vigente. Por mencionar un par de ejemplos
\end{flushleft}


\begin{flushleft}
reales presentes en la ingenieria actual están la construcción de carreteras, la
\end{flushleft}


\begin{flushleft}
realización de un cableado subterráneo o el movimiento de un brazo robótico.
\end{flushleft}


\begin{flushleft}
Este problema también aparece en la teor\i{}́a de la relatividad general: las partćulas
\end{flushleft}


\begin{flushleft}
materiales se mueven a lo largo de geodésicas temporales del espacio-tiempo
\end{flushleft}


\begin{flushleft}
curvo.
\end{flushleft}


\begin{flushleft}
El primer registro matemático vinculado a este problema es de 1697 cuando
\end{flushleft}


\begin{flushleft}
Johann Bernoulli resolvió el problema de la distancia más corta entre dos puntos
\end{flushleft}


\begin{flushleft}
en una superficie convexa, y demostró que el plano osculador de la geodésica debe
\end{flushleft}


\begin{flushleft}
ser perpendicular al plano tangente de la misma.
\end{flushleft}


\begin{flushleft}
Más tarde Euler, en 1732, obtuvo las ecuaciones impl\i{}́citas de las geodésicas.
\end{flushleft}


\begin{flushleft}
Posteriormente Gilbert Bliss obtuvo las l\i{}́neas geodésicas en el anillo de anclaje
\end{flushleft}


\begin{flushleft}
con forma de toroide. Munchmeyer y Haw aplicaron las curvas geodésicas al
\end{flushleft}


\begin{flushleft}
diseño de cascos de embarcaciones.Beck et al. resolvieron el problema de dos
\end{flushleft}


\begin{flushleft}
valores iniciales de la curva geodésica, usando el método de cuarto orden de
\end{flushleft}


\begin{flushleft}
Runge-Kutta en un spline bicúbico. Patrikalakis y Bardis calcularon los
\end{flushleft}


\begin{flushleft}
desplazamientos geodésicos de curvas en superficies B-spline racionales
\end{flushleft}


\begin{flushleft}
utilizando la integración de valores iniciales de geodésicas a una curva generatriz
\end{flushleft}


\begin{flushleft}
en la superficie.
\end{flushleft}


\begin{flushleft}
Sneyd y Peskin investigaron el cálculo de trayectos geodésicos en un cilindro
\end{flushleft}


\begin{flushleft}
generalizado basado en un problema de valor inicial usando un método de
\end{flushleft}


\begin{flushleft}
Runge-Kutta de segundo orden.Kimmel et al. presentó un método numérico para
\end{flushleft}


\begin{flushleft}
encontrar el camino más corto en la superficie mediante el cálculo de la
\end{flushleft}


\begin{flushleft}
propagación de un contorno de igual distancia geodésica desde un punto o una
\end{flushleft}


\begin{flushleft}
región fuente en la superficie. El algoritmo funciona en una cuadricula rectangular
\end{flushleft}


\begin{flushleft}
usando aproximaciones de diferencias finitas.
\end{flushleft}


\begin{flushleft}
Maekawa y Robinson y Armstrong calcularon las geodésicas discretizando las
\end{flushleft}


\begin{flushleft}
ecuaciones diferenciales que rigen la curva geodésca utilizando una aproximación
\end{flushleft}


\begin{flushleft}
de diferencia finita en una malla de puntos, lo que reduce el problema a un
\end{flushleft}


\begin{flushleft}
conjunto de ecuaciones no lineales. El conjunto de las ecuaciones no lineales se
\end{flushleft}


\begin{flushleft}
pueden resolver mediante el método de Newton.
\end{flushleft}





\begin{flushleft}
\newpage
El documento está organizado de la siguiente manera:
\end{flushleft}


$\bullet$





\begin{flushleft}
La Sección 2 define formalmente el problema de estudio.
\end{flushleft}





$\bullet$





\begin{flushleft}
La Sección 3 presenta la metodolog\i{}́a empleada para su resolución.
\end{flushleft}





$\bullet$





\begin{flushleft}
La Sección 4 desarrolla un análisis del rendimiento del método de Euler hacia
\end{flushleft}


\begin{flushleft}
adelante.
\end{flushleft}





$\bullet$





\begin{flushleft}
La Sección 5, finalmente, se presenta las principales conclusiones de este trabajo,
\end{flushleft}


\begin{flushleft}
los elementos aprendidos y posibles l\i{}́neas de trabajo futuro.
\end{flushleft}





\newpage
2





2.1





\begin{flushleft}
Problema
\end{flushleft}





\begin{flushleft}
Ecuaciones diferenciales que rigen una geodésica
\end{flushleft}





\begin{flushleft}
En el caso de una superficie plana, hallar la m\i{}́nima distancia, es trivial: es una
\end{flushleft}


\begin{flushleft}
l\i{}́nea recta desde el punto de partida en dirección al de llegada. En el caso de
\end{flushleft}


\begin{flushleft}
superficies curvas el análisis pasa a ser y más complejo. En este trabajo se
\end{flushleft}


\begin{flushleft}
realizrá, mediante la geometr\i{}́a de Riemann. Los espacios de Riemann son
\end{flushleft}


\begin{flushleft}
aquellos espacios medibles.
\end{flushleft}


\begin{flushleft}
El concepto de geodésica es la generalización de las rectas en una geometr\i{}́a
\end{flushleft}


\begin{flushleft}
plana.
\end{flushleft}


\begin{flushleft}
Aunque las variedades de Riemann son generalmente ''curvas'', no obstante,
\end{flushleft}


\begin{flushleft}
puede encontrarse que dados dos puntos diferentes y suficientemente cercanos
\end{flushleft}


\begin{flushleft}
existe una curva de longitud m\i{}́nima (que no necesariamente es única). Estas
\end{flushleft}


\begin{flushleft}
l\i{}́neas de m\i{}́nima longitud, se llaman l\i{}́neas geodésicas y son una generalización
\end{flushleft}


\begin{flushleft}
del concepto ''l\i{}́nea recta'' o ''l\i{}́nea de m\i{}́nima longitud''. Son las curvas que
\end{flushleft}


\begin{flushleft}
localmente conectan sus puntos a lo largo de las trayectorias más cortas.
\end{flushleft}


\begin{flushleft}
El análisis se hará bajo superficies de Riemann y de las curvas, solamente se
\end{flushleft}


\begin{flushleft}
requiere que en cada uno de sus puntos, haya una métrica eucl\i{}́dea definida sobre
\end{flushleft}


\begin{flushleft}
el espacio tangente, que cambie suavemente de punto a punto. Cuando una curva,
\end{flushleft}


\begin{flushleft}
en una superficie, tenga la propiedad de minimizar la longitudes de las curvas que
\end{flushleft}


\begin{flushleft}
unen dos puntos determinados, se la llamará geodésica minimizante.
\end{flushleft}


\begin{flushleft}
Previo al estudio de las curvas geodésicas en particular, se definirán
\end{flushleft}


\begin{flushleft}
propiedades de una superficie.
\end{flushleft}


\begin{flushleft}
Cuando se quiere cuantificar la curvatura de una superficie S en cierto punto,
\end{flushleft}


\begin{flushleft}
se considera una curva C de dicha superficie que contenga al punto P en cuestión.
\end{flushleft}


\begin{flushleft}
Los vectores unitarios t (tangente) y n (normal) a la curva C en P, están
\end{flushleft}


\begin{flushleft}
relacionados en la figura 1.
\end{flushleft}





\begin{flushleft}
\newpage
Fig. 1. Definición del Vector Normal a una Curva
\end{flushleft}





\begin{flushleft}
MODIFICAR ESTO:
\end{flushleft}


\begin{flushleft}
k=
\end{flushleft}





\begin{flushleft}
dt
\end{flushleft}


\begin{flushleft}
= $\kappa$n = kn + k1 ,
\end{flushleft}


\begin{flushleft}
ds
\end{flushleft}





(1)





\begin{flushleft}
Donde kn es el vector normal de curvatura, y k1 es el vector de curvatura
\end{flushleft}


\begin{flushleft}
geodesica. Este par de vectores, son las componentes del vector de curvatura k de
\end{flushleft}


\begin{flushleft}
C. De esta forma, el vector normal de curvatura, puede expresarse como:
\end{flushleft}


\begin{flushleft}
kn = $\kappa$n N
\end{flushleft}





(2)





\begin{flushleft}
siendo $\kappa$n el valor de la curvatura de la superficie en P en la dirección de t. En
\end{flushleft}


\begin{flushleft}
otras palabras, $\kappa$n es la magnitud de la proyección de k en la superficie normal a P,
\end{flushleft}


\begin{flushleft}
con el signo determinado por la orientación de la superficie normal a P.
\end{flushleft}


\begin{flushleft}
Siguiendo la definicion del libro sugerido [1] en este trabajo se a definen a las
\end{flushleft}


\begin{flushleft}
curvas geodesicas, como aquellas que su curvatura geodesica es 0.
\end{flushleft}


\begin{flushleft}
Se observa que el plano osculador a una curva geodesica (aquel formado por
\end{flushleft}


\begin{flushleft}
los vectores t y n) contiene la superficie normal. A partir de esto, puede verse con
\end{flushleft}


\begin{flushleft}
facilidad que la geodesica entre dos puntos de una esfera, es siempre un gran
\end{flushleft}


\begin{flushleft}
circulo. El probelma ahora pasa a ser que existen dos arcos entre esos puntos y
\end{flushleft}


\begin{flushleft}
solo uno de ellos proviene de la curva con menor distancia (excepto si los dos
\end{flushleft}


\begin{flushleft}
puntos están diametralmente opuestos en ese circulo).Este caso, indica que es
\end{flushleft}


\begin{flushleft}
posible que existan más de una curva geodesica entre dos puntos en una suerficie.
\end{flushleft}


\begin{flushleft}
Dada la parametrización de una superficie como r = r(u, v) y C un arco
\end{flushleft}





\begin{flushleft}
\newpage
parametrizado que pase por el punto P tal cual se muestra en la figura 1 y denotada
\end{flushleft}


\begin{flushleft}
por:
\end{flushleft}


\begin{flushleft}
r(s) = r(u(s), v(s))
\end{flushleft}





(3)





\begin{flushleft}
Entonces las primeras formas fundamentales se definen como:
\end{flushleft}





\begin{flushleft}
I = ds2 = dr.dr = Edu2 + 2Fdudv + Gdv2
\end{flushleft}





\begin{flushleft}
Donde los coeficientes de la primera f orma f undamental son:
\end{flushleft}


\begin{flushleft}
E = ru .ru
\end{flushleft}


\begin{flushleft}
F = ru .rv
\end{flushleft}


\begin{flushleft}
G = rv .rv
\end{flushleft}





(4)


(5)


(6)





\begin{flushleft}
Siendo t el vector tangente a C en P, n el vector normal a C en P, N el vector
\end{flushleft}


\begin{flushleft}
unitario normal a S en P y u el vector unitario perpendicular a t en el plano tangente
\end{flushleft}


\begin{flushleft}
a la suferficie. Definido u = N x t.La compontente u en el vector de curvatura k de
\end{flushleft}


\begin{flushleft}
r(s), es la curvatura geodesica k1 y está dada por la siguiente ecuación.
\end{flushleft}


\begin{flushleft}
k1 = (k.u)u
\end{flushleft}





(7)





\begin{flushleft}
k1 = (k.u)
\end{flushleft}





(8)





\begin{flushleft}
La función escalar
\end{flushleft}





\begin{flushleft}
es llamada curvatura geodésica de C en P. O, equivalentemente:
\end{flushleft}





\begin{flushleft}
k1 =
\end{flushleft}





\begin{flushleft}
dt
\end{flushleft}


\begin{flushleft}
.(Nxt)
\end{flushleft}


\begin{flushleft}
ds
\end{flushleft}





(9)





\begin{flushleft}
El vector tangente unitario, a la curva C puede obtenerse derivando la ecuacion
\end{flushleft}


\begin{flushleft}
3 respecto al arco, usando la siguiente relacion:
\end{flushleft}





\begin{flushleft}
\newpage
t=
\end{flushleft}





\begin{flushleft}
dr(u(s), v(d))
\end{flushleft}


\begin{flushleft}
du
\end{flushleft}


\begin{flushleft}
dv
\end{flushleft}


\begin{flushleft}
= ru
\end{flushleft}


\begin{flushleft}
+ rv
\end{flushleft}


\begin{flushleft}
ds
\end{flushleft}


\begin{flushleft}
ds
\end{flushleft}


\begin{flushleft}
ds
\end{flushleft}





(10)





\begin{flushleft}
Por lo que se obtiene:
\end{flushleft}


\begin{flushleft}
dt
\end{flushleft}


\begin{flushleft}
du
\end{flushleft}


\begin{flushleft}
du dv
\end{flushleft}


\begin{flushleft}
dv
\end{flushleft}


\begin{flushleft}
d2 u
\end{flushleft}


\begin{flushleft}
d2 v
\end{flushleft}


\begin{flushleft}
= ruu ( )2 + 2ruv
\end{flushleft}


\begin{flushleft}
+ rvv ( )2 + ru 2 + rv 2
\end{flushleft}


\begin{flushleft}
ds
\end{flushleft}


\begin{flushleft}
ds
\end{flushleft}


\begin{flushleft}
ds ds
\end{flushleft}


\begin{flushleft}
ds
\end{flushleft}


\begin{flushleft}
ds
\end{flushleft}


\begin{flushleft}
ds
\end{flushleft}





(11)





\begin{flushleft}
en este sentido, sustituyendo en 9 los resultado de 10 y 11:
\end{flushleft}


\begin{flushleft}
du 3
\end{flushleft}


\begin{flushleft}
du dv
\end{flushleft}


\begin{flushleft}
) + (2ru xruv + rv + ruu )( )2 +
\end{flushleft}


\begin{flushleft}
ds
\end{flushleft}


\begin{flushleft}
ds ds
\end{flushleft}


\begin{flushleft}
du dv 2
\end{flushleft}


\begin{flushleft}
du 3
\end{flushleft}


\begin{flushleft}
(ru xrvv + 2rv + ruv ) ( ) + (rv xrvv )( ) ].N+
\end{flushleft}


\begin{flushleft}
ds ds
\end{flushleft}


\begin{flushleft}
ds
\end{flushleft}


\begin{flushleft}
du d2 v d2 u dv
\end{flushleft}


\begin{flushleft}
(ru xrv ).N(
\end{flushleft}


$-$


)


\begin{flushleft}
ds ds2 ds2 ds
\end{flushleft}





\begin{flushleft}
k1 = [(ru xruu )(
\end{flushleft}





(12)





2





\begin{flushleft}
d v
\end{flushleft}


\begin{flushleft}
Paraboloide Se observa que los coeficientes de ( du
\end{flushleft}


\begin{flushleft}
)3 , ( du
\end{flushleft}


\begin{flushleft}
)2 dv
\end{flushleft}


\begin{flushleft}
, du ( dv )2 , ( dv
\end{flushleft}


\begin{flushleft}
)3 , ( du
\end{flushleft}


$-$


\begin{flushleft}
ds
\end{flushleft}


\begin{flushleft}
ds
\end{flushleft}


\begin{flushleft}
ds ds ds
\end{flushleft}


\begin{flushleft}
ds
\end{flushleft}


\begin{flushleft}
ds ds2
\end{flushleft}


\begin{flushleft}
d2 u dv
\end{flushleft}


\begin{flushleft}
) son todos funciones de los coeficientes definidos en 4-6 y de sus derivadas
\end{flushleft}


\begin{flushleft}
ds2 ds
\end{flushleft}


\begin{flushleft}
redpecto a u y v. Es interesante notar que la curvatura geodesica depende solamente
\end{flushleft}


\begin{flushleft}
de la primera forma fundamental.
\end{flushleft}


\begin{flushleft}
Usando los s\i{}́mbolos de Christoffel $\Gamma$ijk (i, j, k = 1, 2) definidos de la siguiente
\end{flushleft}


\begin{flushleft}
forma:
\end{flushleft}





\begin{flushleft}
GEu $-$ 2FFu + FEv
\end{flushleft}


\begin{flushleft}
2(EG $-$ F2 )
\end{flushleft}


\begin{flushleft}
2EFu $-$ EEv + FEu
\end{flushleft}


\begin{flushleft}
$\Gamma$211 =
\end{flushleft}


\begin{flushleft}
2(EG $-$ F2 )
\end{flushleft}


\begin{flushleft}
GEv $-$ FGu
\end{flushleft}


\begin{flushleft}
$\Gamma$112 =
\end{flushleft}


\begin{flushleft}
2(EG $-$ F2 )
\end{flushleft}


\begin{flushleft}
EGu $-$ FEv
\end{flushleft}


\begin{flushleft}
$\Gamma$212 =
\end{flushleft}


\begin{flushleft}
2(EG $-$ F2 )
\end{flushleft}


\begin{flushleft}
2GFv $-$ GGu + FGv
\end{flushleft}


\begin{flushleft}
$\Gamma$122 =
\end{flushleft}


\begin{flushleft}
2(EG $-$ F2 )
\end{flushleft}


\begin{flushleft}
EGu $-$ FEv
\end{flushleft}


\begin{flushleft}
$\Gamma$222 =
\end{flushleft}


\begin{flushleft}
2(EG $-$ F2 )
\end{flushleft}


\begin{flushleft}
$\Gamma$111 =
\end{flushleft}





(13)


(14)


(15)


(16)


(17)


(18)





\begin{flushleft}
\newpage
la curvatura geodésica se reduce a:
\end{flushleft}


\begin{flushleft}
du 3
\end{flushleft}


\begin{flushleft}
du dv
\end{flushleft}


\begin{flushleft}
) + (2$\Gamma$212 $-$ $\Gamma$111 )( )2 +
\end{flushleft}


\begin{flushleft}
ds
\end{flushleft}


\begin{flushleft}
ds ds
\end{flushleft}


\begin{flushleft}
dv
\end{flushleft}


\begin{flushleft}
du
\end{flushleft}


\begin{flushleft}
dv
\end{flushleft}


\begin{flushleft}
($\Gamma$222 $-$ 2$\Gamma$112 ) ( )2 $-$ $\Gamma$122 ( )3 +
\end{flushleft}


\begin{flushleft}
ds ds
\end{flushleft}


\begin{flushleft}
ds
\end{flushleft}


\begin{flushleft}
du d2 v d2 u dv p
\end{flushleft}


$-$


\begin{flushleft}
] (EG $-$ F2 )
\end{flushleft}


\begin{flushleft}
ds ds2 ds2 ds
\end{flushleft}





\begin{flushleft}
k1 = [$\Gamma$211 (
\end{flushleft}





(19)





\begin{flushleft}
Acorde a la definición, se puede determinar la ecuación diferencial de calquier
\end{flushleft}


\begin{flushleft}
geodésica de cualquier superficie simplemente exigiendo que el valor k1 = 0. Al
\end{flushleft}


\begin{flushleft}
imponer esto en la ecuación 20 se obtiene:
\end{flushleft}


\begin{flushleft}
du 2 dv
\end{flushleft}


\begin{flushleft}
du d2 v d2 u dv
\end{flushleft}


\begin{flushleft}
2 du 3
\end{flushleft}


2


1


1


\begin{flushleft}
2 du dv 2
\end{flushleft}


\begin{flushleft}
1 dv 3
\end{flushleft}


=


\begin{flushleft}
$-$$\Gamma$
\end{flushleft}


(


)


$-$


\begin{flushleft}
(2$\Gamma$
\end{flushleft}


$-$


\begin{flushleft}
$\Gamma$
\end{flushleft}


)(


)


+


\begin{flushleft}
(2$\Gamma$
\end{flushleft}


$-$


\begin{flushleft}
$\Gamma$
\end{flushleft}


)


(


)


+


\begin{flushleft}
$\Gamma$
\end{flushleft}


)


$-$


22


22 (


11


12


11


12


\begin{flushleft}
ds ds2 ds2 ds
\end{flushleft}


\begin{flushleft}
ds
\end{flushleft}


\begin{flushleft}
ds ds
\end{flushleft}


\begin{flushleft}
ds ds
\end{flushleft}


\begin{flushleft}
ds
\end{flushleft}


(20)


\begin{flushleft}
Alternativamente, se puede derivar la ecuacion diferencial de la geodesica
\end{flushleft}


\begin{flushleft}
considerando que la superficie normal N tiene la direccion normal a la curva
\end{flushleft}


\begin{flushleft}
geodesica + $-$ n.
\end{flushleft}


\begin{flushleft}
nru = 0, nrv = 0
\end{flushleft}





(21)





.


\begin{flushleft}
Mientras kn =
\end{flushleft}





\begin{flushleft}
dt
\end{flushleft}


,


\begin{flushleft}
ds
\end{flushleft}





\begin{flushleft}
la ecuacion 21 puede ser escrita de la forma:
\end{flushleft}


\begin{flushleft}
dt
\end{flushleft}


\begin{flushleft}
dt
\end{flushleft}


\begin{flushleft}
ru = 0, rv
\end{flushleft}


\begin{flushleft}
ds
\end{flushleft}


\begin{flushleft}
ds
\end{flushleft}





=0





(22)





\begin{flushleft}
Si se sustituye en la ecuacion 22 en la 11 se tiene:
\end{flushleft}


\begin{flushleft}
du 2
\end{flushleft}


\begin{flushleft}
d2 v
\end{flushleft}


\begin{flushleft}
du dv
\end{flushleft}


\begin{flushleft}
dv 2
\end{flushleft}


\begin{flushleft}
d2 u
\end{flushleft}


\begin{flushleft}
(ruu ru )( ) + 2(ruv ru )
\end{flushleft}


\begin{flushleft}
+ (rvv ru )( ) + E 2 + F 2 = 0
\end{flushleft}


\begin{flushleft}
ds
\end{flushleft}


\begin{flushleft}
ds ds
\end{flushleft}


\begin{flushleft}
ds
\end{flushleft}


\begin{flushleft}
ds
\end{flushleft}


\begin{flushleft}
ds
\end{flushleft}





(23)





\begin{flushleft}
du 2
\end{flushleft}


\begin{flushleft}
du dv
\end{flushleft}


\begin{flushleft}
dv
\end{flushleft}


\begin{flushleft}
d2 u
\end{flushleft}


\begin{flushleft}
d2 v
\end{flushleft}


\begin{flushleft}
) + 2(ruv rv )
\end{flushleft}


\begin{flushleft}
+ (rvv rv )( )2 + F 2 + G 2 = 0
\end{flushleft}


\begin{flushleft}
ds
\end{flushleft}


\begin{flushleft}
ds ds
\end{flushleft}


\begin{flushleft}
ds
\end{flushleft}


\begin{flushleft}
ds
\end{flushleft}


\begin{flushleft}
ds
\end{flushleft}





(24)





\begin{flushleft}
(ruu rv )(
\end{flushleft}





\newpage
2





\begin{flushleft}
Al eliminar el término ddsv2 de 2.1, usando 2.1, y eliminando en usando ,
\end{flushleft}


\begin{flushleft}
empleando los s\i{}́mbolos de Christoffel, se obtiene:
\end{flushleft}


\begin{flushleft}
du
\end{flushleft}


\begin{flushleft}
= p,
\end{flushleft}


\begin{flushleft}
ds
\end{flushleft}


\begin{flushleft}
dv
\end{flushleft}


\begin{flushleft}
= q,
\end{flushleft}


\begin{flushleft}
ds
\end{flushleft}





(25)


(26)





\begin{flushleft}
Las ecuaciones y estám relacionanadas por la primera forma fundamental, ds2 =
\end{flushleft}


\begin{flushleft}
Eds2 + 2Fdudv + Gdv2 . Si se elimina el termino ds de ambas ecuaciones, estas
\end{flushleft}


\begin{flushleft}
ecuaciones se reducen a la 20
\end{flushleft}


\begin{flushleft}
du
\end{flushleft}


\begin{flushleft}
ds
\end{flushleft}


\begin{flushleft}
dv
\end{flushleft}


\begin{flushleft}
ds
\end{flushleft}


\begin{flushleft}
dp
\end{flushleft}


\begin{flushleft}
ds
\end{flushleft}


\begin{flushleft}
dp
\end{flushleft}


\begin{flushleft}
ds
\end{flushleft}





2.2





\begin{flushleft}
= p,
\end{flushleft}





(27)





\begin{flushleft}
= q,
\end{flushleft}





(28)





\begin{flushleft}
= $-$$\Gamma$111 p2 $-$ 2$\Gamma$112 pq $-$ $\Gamma$122 q2 ,
\end{flushleft}





(29)





\begin{flushleft}
= $-$$\Gamma$211 p2 $-$ 2$\Gamma$212 pq $-$ $\Gamma$222 q2 .
\end{flushleft}





(30)





\begin{flushleft}
Cálculo de las ecuaciones en superficie paramétrica
\end{flushleft}





\begin{flushleft}
En pos de reforzar el entendimiento del lector sobre el problema se presentan
\end{flushleft}


\begin{flushleft}
algunos ejemplos de superficies y sus respectivos sistemas de ecuaciones
\end{flushleft}


\begin{flushleft}
diferenciales que determinan las curvas geodésicas.
\end{flushleft}





\begin{flushleft}
2.2.1 Plano por el origen
\end{flushleft}


\begin{flushleft}
Sean V1 = (x1 , y1 , z1 ) y V2 = (x2 , y2 , z2 ) los vectores que generan el plano. La
\end{flushleft}


\begin{flushleft}
ecuación paramétrica 31 representa dicho plano.
\end{flushleft}


\begin{flushleft}
r(u, v) = (x1 u + x2 v, y1 u + y2 v, z1 u + z2 v)
\end{flushleft}





(31)





\begin{flushleft}
Utilizando las ecuaciones de la geodésica 27-30, las ecuaciones de la primera
\end{flushleft}


\begin{flushleft}
forma fundamental 4-6 y la ecuación parametrica 31 del plano obtenemos E, F y G
\end{flushleft}


\begin{flushleft}
en 32 y en 33-34 sus respectivas derivadas parciales.
\end{flushleft}





\begin{flushleft}
\newpage
E = x21 + y21 + z21 ,
\end{flushleft}


\begin{flushleft}
Eu = 0,
\end{flushleft}


\begin{flushleft}
Ev = 0,
\end{flushleft}





\begin{flushleft}
F = x1 x2 + y1 y2 + z1 z2 ,
\end{flushleft}


\begin{flushleft}
Fu = 0,
\end{flushleft}


\begin{flushleft}
Fv = 0,
\end{flushleft}





\begin{flushleft}
G = x22 + y22 + z22 ,
\end{flushleft}


\begin{flushleft}
Gu = 0,
\end{flushleft}


\begin{flushleft}
Gv = 0,
\end{flushleft}





(32)


(33)


(34)





\begin{flushleft}
Entonces, como todas las derivadas parciales son nulas, usando 13-18, los
\end{flushleft}


\begin{flushleft}
s\i{}́mbolos de Christoffel valen cero.
\end{flushleft}


\begin{flushleft}
Finalmente obtenemos las ecuaciones de la geodésica para el plano en 35-38.
\end{flushleft}


\begin{flushleft}
du
\end{flushleft}


\begin{flushleft}
ds
\end{flushleft}


\begin{flushleft}
dv
\end{flushleft}


\begin{flushleft}
ds
\end{flushleft}


\begin{flushleft}
dp
\end{flushleft}


\begin{flushleft}
ds
\end{flushleft}


\begin{flushleft}
dq
\end{flushleft}


\begin{flushleft}
ds
\end{flushleft}





\begin{flushleft}
= p,
\end{flushleft}





(35)





\begin{flushleft}
= q,
\end{flushleft}





(36)





= 0,





(37)





=0





(38)





\begin{flushleft}
2.2.2 Esfera unitaria centrada en el origen
\end{flushleft}


\begin{flushleft}
La ecuación paramétrica 39 representa una esfera unitaria centrada en el origen.
\end{flushleft}


\begin{flushleft}
r(u, v) = (cos(u)sin(v), sin(u)sin(v), cos(v)), (u, v) $\in$ [o, 2$\pi$] × [0, $\pi$]
\end{flushleft}





(39)





\begin{flushleft}
De forma análoga que el caso anterior, utilizamos las ecuaciones de la geodésica
\end{flushleft}


\begin{flushleft}
27-30, las ecuaciones de la primera forma fundamental 4-6 y la ecuación
\end{flushleft}


\begin{flushleft}
parametrica 39 del plano obtenemos E, F y G en 40 las ecuaciones de la primera
\end{flushleft}


\begin{flushleft}
forma fundamental y en 41-42 sus respectivas derivadas parciales.
\end{flushleft}


\begin{flushleft}
E = sin2 (v),
\end{flushleft}


\begin{flushleft}
Eu = 0,
\end{flushleft}


\begin{flushleft}
Ev = 2sin(v)cos(v),
\end{flushleft}





\begin{flushleft}
F=0
\end{flushleft}


\begin{flushleft}
Fu = 0,
\end{flushleft}


\begin{flushleft}
Fv = 0,
\end{flushleft}





\begin{flushleft}
G = 1,
\end{flushleft}


\begin{flushleft}
Gu = 0,
\end{flushleft}


\begin{flushleft}
Gv = 0,
\end{flushleft}





(40)


(41)


(42)





\begin{flushleft}
De esta forma, usando 13-18, obtenemos los s\i{}́mbolos de Christoffel en 43-45
\end{flushleft}





\begin{flushleft}
\newpage
$\Gamma$212 = $\Gamma$111 = $\Gamma$122 = $\Gamma$222 = 0,
\end{flushleft}


\begin{flushleft}
cos(v)
\end{flushleft}


,


\begin{flushleft}
$\Gamma$112 =
\end{flushleft}


\begin{flushleft}
sen(v)
\end{flushleft}


\begin{flushleft}
$\Gamma$211 = $-$sen(v)cos(v).
\end{flushleft}





(43)


(44)


(45)





\begin{flushleft}
Las ecuaciones de la geodésica para la esfera son 46-49.
\end{flushleft}


\begin{flushleft}
du
\end{flushleft}


\begin{flushleft}
ds
\end{flushleft}


\begin{flushleft}
dv
\end{flushleft}


\begin{flushleft}
ds
\end{flushleft}


\begin{flushleft}
dp
\end{flushleft}


\begin{flushleft}
ds
\end{flushleft}


\begin{flushleft}
dq
\end{flushleft}


\begin{flushleft}
ds
\end{flushleft}





\begin{flushleft}
= p,
\end{flushleft}





(46)





\begin{flushleft}
= q,
\end{flushleft}





(47)





\begin{flushleft}
$-$2cos(v)
\end{flushleft}


\begin{flushleft}
pq,
\end{flushleft}


\begin{flushleft}
sen(v)
\end{flushleft}





(48)





\begin{flushleft}
= sen(v)cos(v)p2 .
\end{flushleft}





(49)





=





\begin{flushleft}
2.2.3 Paraboloide hiperbólico
\end{flushleft}


\begin{flushleft}
La ecuación paramétrica 50 representa el paraboloide hiperbólico.
\end{flushleft}


\begin{flushleft}
r(u, v) = (u, v, uv)
\end{flushleft}





(50)





\begin{flushleft}
Nuevamente, a partir de las ecuaciones de la geodésica 27-30, las ecuaciones
\end{flushleft}


\begin{flushleft}
de la primera forma fundamental 4-6 y la ecuación parametrica 50 del plano
\end{flushleft}


\begin{flushleft}
obtenemos E, F y G en 51 las ecuaciones de la primera forma fundamental y en
\end{flushleft}


\begin{flushleft}
52-53 sus respectivas derivadas parciales.
\end{flushleft}


\begin{flushleft}
E = 1 + v2 ,
\end{flushleft}


\begin{flushleft}
Eu = 0,
\end{flushleft}


\begin{flushleft}
Ev = 2v,
\end{flushleft}





\begin{flushleft}
F = uv
\end{flushleft}


\begin{flushleft}
Fu = v,
\end{flushleft}


\begin{flushleft}
Fv = u,
\end{flushleft}





\begin{flushleft}
G = 1 + u2 ,
\end{flushleft}


\begin{flushleft}
Gu = 2u,
\end{flushleft}


\begin{flushleft}
Gv = 0,
\end{flushleft}





(51)


(52)


(53)





\begin{flushleft}
A partir de lo anterior se sustituye en 13-18 y se obtienen los s\i{}́mbolos de
\end{flushleft}


\begin{flushleft}
Christoffel en 54-56.
\end{flushleft}





\begin{flushleft}
\newpage
$\Gamma$111 = $\Gamma$211 = $\Gamma$122 = $\Gamma$222 = 0,
\end{flushleft}


\begin{flushleft}
v
\end{flushleft}


,


\begin{flushleft}
$\Gamma$112 = 2
\end{flushleft}


\begin{flushleft}
u + v2 + 1
\end{flushleft}


\begin{flushleft}
u
\end{flushleft}


\begin{flushleft}
$\Gamma$212 = 2
\end{flushleft}


.


\begin{flushleft}
u + v2 + 1
\end{flushleft}





(54)


(55)


(56)





\begin{flushleft}
Las ecuaciones de la geodésica para el paraboloide hiperbólico son 57-60.
\end{flushleft}


\begin{flushleft}
du
\end{flushleft}


\begin{flushleft}
ds
\end{flushleft}


\begin{flushleft}
dv
\end{flushleft}


\begin{flushleft}
ds
\end{flushleft}


\begin{flushleft}
dp
\end{flushleft}


\begin{flushleft}
ds
\end{flushleft}


\begin{flushleft}
dq
\end{flushleft}


\begin{flushleft}
ds
\end{flushleft}





\begin{flushleft}
= p,
\end{flushleft}





(57)





\begin{flushleft}
= q,
\end{flushleft}





(58)





\begin{flushleft}
$-$2v
\end{flushleft}


\begin{flushleft}
pq,
\end{flushleft}


\begin{flushleft}
+ v2 + 1
\end{flushleft}


\begin{flushleft}
$-$2u
\end{flushleft}


= 2


\begin{flushleft}
pq.
\end{flushleft}


\begin{flushleft}
u + v2 + 1
\end{flushleft}


=





\begin{flushleft}
u2
\end{flushleft}





(59)


(60)





\newpage
3





\begin{flushleft}
Metodolog\i{}́a
\end{flushleft}





\begin{flushleft}
El sistema de ecuaciones diferenciales 27-30 puede resolverse modelándolo de las
\end{flushleft}


\begin{flushleft}
siguientes dos formas:
\end{flushleft}


$\bullet$





\begin{flushleft}
Problema de Valor Inicial (PVI).
\end{flushleft}





$\bullet$





\begin{flushleft}
Problema con Condiciones de Borde (PCB).
\end{flushleft}





\begin{flushleft}
La primera forma consiste en proporcionar el valor inicial para cada una de las
\end{flushleft}


\begin{flushleft}
cuatro ecuaciones diferenciales del sistema. Como la solución de un PVI es única,
\end{flushleft}


\begin{flushleft}
al indicar los valores iniciales queda determinada una sola curva.
\end{flushleft}


\begin{flushleft}
En el caso del PCB se dan dos puntos del espacio como condiciones de borde por
\end{flushleft}


\begin{flushleft}
lo que las ecuaciones diferenciales del sistema tienen muchas soluciones o incluso
\end{flushleft}


\begin{flushleft}
ninguna.
\end{flushleft}


\begin{flushleft}
Generalmente problema a resolver se asemeja más a un PCB que a un PVI, por
\end{flushleft}


\begin{flushleft}
lo que la forma más natural de resolverlo ser\i{}́a a partir del segundo. Debido a que
\end{flushleft}


\begin{flushleft}
la complejidad computacional que implica el PCB es considerablemente mayor
\end{flushleft}


\begin{flushleft}
que el PVI se optó por el primer método para la resolución. En 3.1 se demuestra
\end{flushleft}


\begin{flushleft}
que es posible modelar las geodésicas como un PVI.
\end{flushleft}


\begin{flushleft}
Por otro lado el sistema presentado en 27-30 es un sistema anal\i{}́tico y no
\end{flushleft}


\begin{flushleft}
puede resolverse computacionalmente por lo que es necesario utilizar un
\end{flushleft}


\begin{flushleft}
algoritmo numérico para para modelar y aproximar el problema. El método
\end{flushleft}


\begin{flushleft}
seleccionado para el estudio del fue el método de Euler hacia adelante, que fue
\end{flushleft}


\begin{flushleft}
elegido debido a su simplicidad en comparación a otros métodos considerados
\end{flushleft}


\begin{flushleft}
como el Método del Trapecio.
\end{flushleft}


3.1





\begin{flushleft}
Problema de Valor Inicial (PVI)
\end{flushleft}





\begin{flushleft}
El Problema de Valor Inicial, también conocido como problema de Cauchy, es
\end{flushleft}


\begin{flushleft}
básicamente un sistema de ecuaciones compuesto de una EDO y una condición
\end{flushleft}


\begin{flushleft}
inicial de la misma.
\end{flushleft}


\begin{flushleft}
A continuación se explica en más detalle y demuestra que las ecuaciones que
\end{flushleft}


\begin{flushleft}
rigen las geodésicas se pueden considerar un PVI.
\end{flushleft}


\begin{flushleft}
Previamente, las siguientes definiciones REFCURSO nos ayudaran en la
\end{flushleft}


\begin{flushleft}
muestra:
\end{flushleft}


$\bullet$





\begin{flushleft}
Una ecuación diferencial es una ecuación que relaciona las derivadas de una o
\end{flushleft}





\begin{flushleft}
\newpage
más variables dependientes respecto a una o más variables independientes.
\end{flushleft}


$\bullet$





\begin{flushleft}
Una ecuación diferencial ordinaria (EDO) es una ecuación diferencial que
\end{flushleft}


\begin{flushleft}
relaciona una función desconocida de una única variable independiente con sus
\end{flushleft}


\begin{flushleft}
derivadas.
\end{flushleft}





$\bullet$





\begin{flushleft}
Dada una función f : R2 $\rightarrow$ R el Problema de Valores Iniciales consiste en hallar
\end{flushleft}


\begin{flushleft}
la función y = y(x) tal que:
\end{flushleft}


\begin{flushleft}

\end{flushleft}


\begin{flushleft}

\end{flushleft}


0


\begin{flushleft}

\end{flushleft}


\begin{flushleft}

\end{flushleft}


\begin{flushleft}
 y (x) = f (x, y(x))
\end{flushleft}


\begin{flushleft}
(PVI) : 
\end{flushleft}


\begin{flushleft}

\end{flushleft}


\begin{flushleft}

\end{flushleft}


\begin{flushleft}
 y(x0 ) = y(y0 ) $\in$ R
\end{flushleft}





\begin{flushleft}
Entonces utilizando las ecuaciones planteadas en 27 - 30 y considerando la
\end{flushleft}


\begin{flushleft}
siguiente notación:
\end{flushleft}


\begin{flushleft}
du
\end{flushleft}


\begin{flushleft}
ds
\end{flushleft}


\begin{flushleft}
dv
\end{flushleft}


\begin{flushleft}
v0 (s) =
\end{flushleft}


\begin{flushleft}
ds
\end{flushleft}


\begin{flushleft}
dp
\end{flushleft}


\begin{flushleft}
p0 (s) =
\end{flushleft}


\begin{flushleft}
ds
\end{flushleft}


\begin{flushleft}
dq
\end{flushleft}


\begin{flushleft}
q0 (s) =
\end{flushleft}


\begin{flushleft}
ds
\end{flushleft}





\begin{flushleft}
u0 (s) =
\end{flushleft}





\begin{flushleft}
=p
\end{flushleft}





\begin{flushleft}
$\rightarrow$ u0 (s) = f1 (s, u(s)),
\end{flushleft}





(61)





\begin{flushleft}
=q
\end{flushleft}





\begin{flushleft}
$\rightarrow$ v0 (s) = f2 (s, v(s)),
\end{flushleft}





(62)





\begin{flushleft}
= $-$$\Gamma$111 p2 $-$ 2$\Gamma$112 pq $-$ $\Gamma$122 q2
\end{flushleft}





\begin{flushleft}
$\rightarrow$ p0 (s) = f3 (s, p(s)),
\end{flushleft}





(63)





\begin{flushleft}
= $-$$\Gamma$211 p2 $-$ 2$\Gamma$212 pq $-$ $\Gamma$222 q2 .
\end{flushleft}





\begin{flushleft}
$\rightarrow$ q0 (s) = f4 (s, p(s)).
\end{flushleft}





(64)





\begin{flushleft}
se puede representar las ecuaciones diferenciales como un PVI de la siguiente
\end{flushleft}


\begin{flushleft}
manera:
\end{flushleft}


\begin{flushleft}

\end{flushleft}


\begin{flushleft}

\end{flushleft}


\begin{flushleft}

\end{flushleft}


\begin{flushleft}

\end{flushleft}


\begin{flushleft}
y0 (s) = f (s, u(s), v(s), p(s), q(s))
\end{flushleft}


\begin{flushleft}

\end{flushleft}


\begin{flushleft}
(PVI) : 
\end{flushleft}


\begin{flushleft}

\end{flushleft}


\begin{flushleft}

\end{flushleft}


\begin{flushleft}
 y(s ) = (u(s ), v(s ), p(s ), q(s )) $\in$ R4
\end{flushleft}


0





0





0





0





0





\begin{flushleft}
Como los ejemplos de la sección 2.2 son casos particulares de las ecuaciones 27-30
\end{flushleft}


\begin{flushleft}
también son PVI, en particular el paraboloide hiperbólico.
\end{flushleft}


3.2





\begin{flushleft}
Discretización del problema
\end{flushleft}





\begin{flushleft}
El método de Euler, denominado as\i{}́ en honor a su autor Leonhard Euler, es un
\end{flushleft}


\begin{flushleft}
algoritmo matemático empleado para la resolución del Problema de Valor Inicial.
\end{flushleft}


\begin{flushleft}
Por su simplicidad este método se utiliza como base para métodos ms complejos y
\end{flushleft}


\begin{flushleft}
de mayor presición.
\end{flushleft}


\begin{flushleft}
Es un mtodo de primer orden, lo cual implica que el error local es proporcional al
\end{flushleft}


\begin{flushleft}
cuadrado del paso, y el error global es proporcional al paso. Tiene dos variantes:
\end{flushleft}





\begin{flushleft}
\newpage
Euler hacia Adelante y Euler hacia Atrás.
\end{flushleft}


\begin{flushleft}
En este caso limitamos nuestro análisis al paraboloide hiperbólico, pero el
\end{flushleft}


\begin{flushleft}
razonamiento es análogo en las otras superficies.
\end{flushleft}


\begin{flushleft}
Partiendo de las ecuaciones (57-60) y utilizando Euler hacia Adelante, se calcula a
\end{flushleft}


\begin{flushleft}
continuación las ecuaciones en su forma discreta.
\end{flushleft}


\begin{flushleft}
Para u:
\end{flushleft}


\begin{flushleft}
uk+1 $-$ uk
\end{flushleft}


\begin{flushleft}
du
\end{flushleft}


\begin{flushleft}
=p$\Rightarrow$
\end{flushleft}


\begin{flushleft}
= pk $\Rightarrow$ uk+1 = uk + pk h
\end{flushleft}


\begin{flushleft}
ds
\end{flushleft}


\begin{flushleft}
h
\end{flushleft}


\begin{flushleft}
Para v:
\end{flushleft}


\begin{flushleft}
vk+1 $-$ vk
\end{flushleft}


\begin{flushleft}
dv
\end{flushleft}


\begin{flushleft}
=q$\Rightarrow$
\end{flushleft}


\begin{flushleft}
= qk $\Rightarrow$ vk+1 = vk + qk h
\end{flushleft}


\begin{flushleft}
ds
\end{flushleft}


\begin{flushleft}
h
\end{flushleft}


\begin{flushleft}
Para p:
\end{flushleft}


\begin{flushleft}
dp
\end{flushleft}


\begin{flushleft}
pk+1 $-$ pk
\end{flushleft}


\begin{flushleft}
$-$2v
\end{flushleft}


\begin{flushleft}
$-$2vk
\end{flushleft}


\begin{flushleft}
$-$2vk
\end{flushleft}


= 2


\begin{flushleft}
pq $\Rightarrow$
\end{flushleft}


= 2


\begin{flushleft}
pk qk $\Rightarrow$ pk+1 = 2
\end{flushleft}


\begin{flushleft}
pk qk h+pk
\end{flushleft}


2


2


\begin{flushleft}
ds u + v + 1
\end{flushleft}


\begin{flushleft}
h
\end{flushleft}


\begin{flushleft}
uk + vk + 1
\end{flushleft}


\begin{flushleft}
uk + v2k + 1
\end{flushleft}


\begin{flushleft}
Finalmente, para q:
\end{flushleft}


\begin{flushleft}
dq
\end{flushleft}


\begin{flushleft}
qk+1 $-$ qk
\end{flushleft}


\begin{flushleft}
$-$2uk
\end{flushleft}


\begin{flushleft}
$-$2u
\end{flushleft}


\begin{flushleft}
$-$2uk
\end{flushleft}


\begin{flushleft}
p
\end{flushleft}


\begin{flushleft}
q
\end{flushleft}


$\Rightarrow$


\begin{flushleft}
q
\end{flushleft}


=


\begin{flushleft}
pk qk h+qk
\end{flushleft}


= 2


\begin{flushleft}
pq
\end{flushleft}


$\Rightarrow$


=


\begin{flushleft}
k
\end{flushleft}


\begin{flushleft}
k
\end{flushleft}


\begin{flushleft}
k+1
\end{flushleft}


\begin{flushleft}
ds u + v2 + 1
\end{flushleft}


\begin{flushleft}
h
\end{flushleft}


\begin{flushleft}
u2k + v2k + 1
\end{flushleft}


\begin{flushleft}
u2k + v2k + 1
\end{flushleft}


\begin{flushleft}
En resumen:
\end{flushleft}


\begin{flushleft}
uk+1 = uk + pk h
\end{flushleft}


\begin{flushleft}
vk+1 = vk + qk h
\end{flushleft}


\begin{flushleft}
$-$2vk
\end{flushleft}


\begin{flushleft}
pk+1 = 2
\end{flushleft}


\begin{flushleft}
pk q k h + pk
\end{flushleft}


\begin{flushleft}
uk + v2k + 1
\end{flushleft}


\begin{flushleft}
$-$2uk
\end{flushleft}


\begin{flushleft}
pk qk h + qk
\end{flushleft}


\begin{flushleft}
qk+1 = 2
\end{flushleft}


\begin{flushleft}
uk + v2k + 1
\end{flushleft}





(65)


(66)


(67)


(68)





\begin{flushleft}
El siguiente pseudocódigo 1 expresa una forma de introducir el problema en
\end{flushleft}


\begin{flushleft}
softwares tales como ''Octave'', ''R'', ''Python'', etc.
\end{flushleft}





\begin{flushleft}
\newpage
Algorithm 1 Pseudocódigo para resolver el PVI mediante el método ''Euler hacia
\end{flushleft}


\begin{flushleft}
adelante''
\end{flushleft}


\begin{flushleft}
Require: y(0) = y0 ; x(0) = 0; i = 1; h; f
\end{flushleft}


\begin{flushleft}
1: while x(i) $<$ x f do
\end{flushleft}


2:


\begin{flushleft}
y(i + 1) $\leftarrow$ y(i) + h ∗ f (x(i), y(i))
\end{flushleft}


3:


\begin{flushleft}
x(i + 1) $\leftarrow$ x(i) + h
\end{flushleft}


4:


\begin{flushleft}
i$\leftarrow$i+1
\end{flushleft}


\begin{flushleft}
5: end while
\end{flushleft}





4





\begin{flushleft}
Estudio Experimental
\end{flushleft}





4.1





\begin{flushleft}
Ambiente de trabajo
\end{flushleft}





\begin{flushleft}
Durante el desarrollo de la experiencia se trabajo con:
\end{flushleft}


$\bullet$





\begin{flushleft}
Lenguaje de programación: Octave.
\end{flushleft}





$\bullet$





\begin{flushleft}
Caracter\i{}́sticas de la computadora utilizada:
\end{flushleft}


\begin{flushleft}
· HP Pavilion 15 Notebook
\end{flushleft}


\begin{flushleft}
· Procesador:AMD A10-7300 Radeon R6, 10 Compute Cores 4C+6G
\end{flushleft}


\begin{flushleft}
· Memoria RAM: 8 GB
\end{flushleft}


\begin{flushleft}
· Sistema Operativo: Manjaro 17.1.12 x64 (Deepin desktop)
\end{flushleft}


\begin{flushleft}
· Versión de Octave: 4.4.1
\end{flushleft}





4.2





\begin{flushleft}
Resultados Obtenidos
\end{flushleft}





\begin{flushleft}
En las condiciones del ambiente de trabajo descrito en 4.1 se realizó la ejecución
\end{flushleft}


\begin{flushleft}
de los scripts creados.
\end{flushleft}


\begin{flushleft}
En las figuras 2 - 4 se puede apreciar el resultado obtenido
\end{flushleft}


\begin{flushleft}
Fig. 2. Euler
\end{flushleft}





\begin{flushleft}
\newpage
Fig. 3. Euler
\end{flushleft}





\begin{flushleft}
Fig. 4. Euler
\end{flushleft}





\newpage
5





\begin{flushleft}
Conclusiones
\end{flushleft}





\begin{flushleft}
\newpage
References
\end{flushleft}


\begin{flushleft}
[1] I. Horváth. {``}shape interrogation for computer aided design and manufacturing{``},
\end{flushleft}


\begin{flushleft}
by nicholas m. patrikalakis and takashi maekawa, springer-verlag, berlin, heidelberg,
\end{flushleft}


\begin{flushleft}
new york, 2002, isbn 3-540-42454-7, 408 pages. Structural and Multidisciplinary
\end{flushleft}


\begin{flushleft}
Optimization, 24(6):467--468, Dec 2002.
\end{flushleft}





\begin{flushleft}
Octave:Función Tic() https://octave.sourceforge.io/octave/function/tic.html
\end{flushleft}





\newpage



\end{document}
